\documentclass[12pt]{article}

\usepackage{nopageno}
\usepackage[letterpaper, portrait, margin=1in]{geometry}
\usepackage{newtxtext,newtxmath}
\usepackage{titlesec}
\usepackage{hanging}
\usepackage[table,xcdraw]{xcolor}
\usepackage{tipa}
\usepackage{qtree}


\setlength{\parindent}{0.5in}

\titleformat{\section}
{\normalfont\fontsize{12}{15}\bfseries}{\makebox[0.5in][l]{\thesection}}{0pt}{}
\titleformat{\subsection}
{\normalfont\fontsize{12}{15}\bfseries}{\makebox[0.5in][l]{\thesubsection}}{0pt}{}
\titleformat{\subsubsection}
{\normalfont\fontsize{12}{15}\bfseries}{\makebox[0.5in][l]{\thesubsubsection}}{0pt}{}

\titlespacing*{\section}{0pt}{12pt}{0pt}
\titlespacing*{\subsection}{0pt}{12pt}{0pt}
\titlespacing*{\subsubsection}{0pt}{12pt}{0pt}


\newcounter{data}
\newenvironment{data}{%
	\refstepcounter{data}%
	\setcounter{letter}{0}%
	\bigskip\par\noindent%
	\makebox[0.5in][l]{(\thedata)}%
	\ignorespaces%
}{%
	\ignorespacesafterend%
}

\newcounter{letter}
\newenvironment{letter}{%
	\refstepcounter{letter}%
	\makebox[0.5in][l]{\alph{letter}.}%
	\ignorespaces%
}{%
	\ignorespacesafterend%
}


\newcommand{\titletext}[1]
{\begin{center}
		\textbf{#1}
\end{center}}

\newcommand{\headertext}[4]
{\begin{flushleft}
		#1
		\\ #2
		\\ #3
		\\ #4
\end{flushleft}}

\newenvironment{references}{%
	\begin{hangparas}{0.5in}{1}
	}{%
		\\ \end{hangparas}%
}


\begin{document}
	
	\headertext{Tulasi Holdridge}{Linguistic Analysis}{Dr. Littlefield}{Last Updated: February 2, 2021}
	
	\titletext{Linguistic Analysis Template Demo}
	Each section of this document is correlated to a section of the Linguistic Analysis style sheet. It serves two purposes---first, ``proving'' that this template conforms to the style sheet; second, explaining and demonstrating how to use any features that are LaTeX-specific (such as formatting commands that a user will need to know).
	
	In addition to the sections which directly match the stylesheet, there is a section at the end (before the conclusion) explaining a few miscellaneous LaTeX quirks and tips that might be useful. Throughout the document, if there is no information relevant to formatting and the use of LaTeX the subsection will just say ``N/A.''
	
	\section{Document format}
	These subsections discuss general formatting and the structure of the paper.
	
	\subsection{Page layout}
	The style sheet requires single spaced text and 1-inch margins. Both of these things are automatically enforced by the document---no additional commands required!
	
	\subsection{Font}
	The paper should be in a 12pt serif font. Again, this is automatically enforced, but note that the font in this template is not exactly Times New Roman. Times New Roman is proprietary, so it's more complicated to use it in LaTeX. Since the style sheet says ``a 12pt font like Times New Roman or Doulos SIL,'' I opted for the lookalike instead of actual TNR.
	
	This subsection also specifies that the text in any linguistic trees should match the rest of the document, and trees should be electronically generated. Trees are actually pretty simple in LaTeX (and they're free!) but for concision I describe how to generate trees in section 2.8.
	
	\subsection{Paragraph layout}
	The first paragraph after a heading should not be indented; subsequent paragraphs should. This functionality is provided for you, but starting a new paragraph is a tiny bit different in LaTeX. Instead of pressing \textit{enter} and then \textit{tab}, you will simply press \textit{enter} twice. So, your paragraphs are separated by a blank line in the source file, but in the paper itself they follow the style we want.
	
	\subsection{Section headings} 
	Here's the first LaTeX command you need! Commands are usually a \verb|\| backslash followed by a command name, and then some \verb|{}| curly braces with additional information inside. 
	
	All the formatting details of section headings are handled by this template, including numbering. To create a new section, just use the command \verb|\section{My section title}|. You can check out this document's source code to see exactly how I format it with blank lines, but it actually doesn't even matter if there are blank lines before/after the command. It'll format it correctly regardless. Then, start your next paragraphs and they will conform to the style too.
	
	Creating further subsections is easy too---use \verb|\subsection{My subsection title}| and \verb|\subsubsection{My subsubsection title}|. As mentioned above, the numbering happens automatically, you just need to worry about what your section title will be.
	
	\subsection{Title block}
	Although this also requires LaTeX commands, they've been prefilled for you in the template. All formatting is handled by the commands, which are \verb|\headertext{}{}{}{}| and \verb|\titletext{}|. The curly braces are the only part you need to edit. For the header: the braces should contain your name, the course, the professor's name, and the date (in that order). For the title: the braces should contain the title of the paper.
	
	\subsection{Tables and figures}
	Tables are a little more tricky in LaTeX---I recommend just generating them using an online resource like https://www.tablesgenerator.com.
	
	[more info to come on adding labels to tables/figures and changing font size]
	
	\subsection{Footnotes}
	To add a footnote to your text, use the command \verb|\footnote{}|, putting your citation text in the curly braces. Here's an example.\footnote{Wow, a footnote!} Make sure to put the command exactly where you want it in the text, and no need for extra spaces around it. LaTeX will automatically number your footnotes, similar to the section headings.\footnote{Here's some proof.}
	
	\section{Using and formatting linguistic data}
	This section has an extra subsection at the end, detailing how to generate trees.
	
	\subsection{Small segments of data in the text}
	Small pieces of data like morphemes included in the prose of the paper should be in italics. To italicize something in LaTeX, use the command \verb|\textit{}|, with the italicized text inside the curly braces. Here's a \textit{demonstration}.
	
	\subsection{Longer pieces of data}
	This is a bit trickier. Use \verb|\begin{data}| to create a new numbered piece of data (numbering won't reset within the paper). Then type in your data, but don't leave a blank line under the command unless you want it to start a new paragraph. If you want sub-data to be lettered, use \verb|\begin{letter}| followed by the data you want. Lettering is reset within each numbered data section. Close off both tags with their respective \verb|\end{}| commands. 
	
	If you're going from data back to prose, you'll need to add the command \verb|\bigskip|. Then press enter twice, then use the command \verb|\noindent|, then start writing your next paragraph. If you instead end the section or subsection without going back to prose, you \textit{do not} need these commands. Here's an example of how the data looks (view the .tex file of this PDF to see exactly how I formatted the commands):
	
	\begin{data}
		\begin{letter}
			This is my first piece of data
		\end{letter}
	
		\begin{letter}
			This is my second piece of data.
		\end{letter}
	\end{data}
	\begin{data}
		\begin{letter}
			And a bit more data.
		\end{letter}
	\end{data}
	\begin{data}
		Of course, you can also have data without lettering.
	\end{data}
	\bigskip

	\noindent See the next section for guidance on formatting interlinear glosses, or other data which is aligned in complicated ways.
	
	\subsection{Interlinear glossing}
	Use \verb|\textsc{}| for small caps in your glosses, \textsc{like so}.
	
	\subsection{Glosses}
	See section 6 for an aside about quotation marks in LaTeX.
	
	\subsection{Parentheses}
	N/A
	
	\subsection{Capitalization in examples}
	N/A
	
	\subsection{Phonetic and phonemic material}
	N/A
	
	\subsection{Linguistic trees}
	The template includes the \verb|qtree| package for you, which can be used to create trees.
	Here's a basic example of the command for a tree, and the tree it generates:
	
	\verb|\Tree [.S This [.VP [.V is ] \qroof{a simple tree}.NP ] ]|
	
	\Tree [.S This [.VP [.V is ] \qroof{a simple tree}.NP ] ]
	
	As you can see, the font and text size are automatically correct. The command for the tree looks a bit complicated, but it might help to break it into lines which more clearly show its structure:
	
	\begin{verbatim}
		\Tree [.S This
		          [.VP [.V is ]
		               \qroof{a simple tree}.NP ] ]
	\end{verbatim}

	You can trial-and-error your way through making similar trees, or read the qtree documentation at https://www.ling.upenn.edu/advice/latex/qtree/qtreenotes.pdf for additional help. Just make sure you put a space before each \verb|]| closing bracket, or it'll break.
	
	
	\section{General punctuation}
	This section of the style sheet just describes things you'll have to keep in mind while writing.
	
	\subsection{Quoting outside sources}
	See section 6 for an aside about quotation marks in LaTeX.
	
	\subsection{Commas and periods}
	N/A
	
	\section{List of abbreviations}
	
	\section{References and citations}
	The command \verb|\begin{references}| will create an ``environment'' where every paragraph has hanging indents. At the end of the section, use \verb|\end{references}|. The section is already set up for you in the template---just put your citations between the begin and end tags.
	
	In terms of writing the citations themselves, you'll have to use the style sheet and type the exact words of the citation, making sure to use \verb|\textit{}| to italicize any necessary parts.
	
	\section{Miscellaneous LaTeX things}
	
	\section{Conclusion}
	
\end{document}